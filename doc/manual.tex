\documentclass[a4paper,11pt,titlepage]{article}

\usepackage[utf8]{inputenc}
\usepackage[czech]{babel}
\usepackage[left=2cm,top=3cm,text={17cm,24cm}]{geometry}
\usepackage{graphicx}
\graphicspath{ {images/} }
\usepackage{listings}
\usepackage{url}
\usepackage{amsmath, amsthm, amssymb}

\begin{document}


\begin{titlepage}
	
	\begin{center}
		{\Huge\textsc{Vysoké učení technické v~Brně}}\\
		\medskip
		{\huge\textsc{Fakulta informačních technologií}}\\
		\vspace{\stretch{0.382}}
		{\huge Téma 1: Stomatologická ordinace}\\
		\medskip
		{\LARGE Technická zpráva k projektu do předmětu IMS}\\
		\vspace{\stretch{0.618}}
	\end{center}
	
	Ladislav Bezděčík xbezdec13@stud.fit.vutbr.cz \hfill{} 

	Drahník Lukáš xdrahn00@stud.fit.vutbr.cz \hfill{Brno, \today}
	
\end{titlepage}

\newpage

\tableofcontents

\newpage

\section{Úvod}

Práce se zabývá simulací (viz. \cite{ims}, slajd 8) modelu (viz. \cite{ims}, slajd 7) stomatologické ordinace. Na základě modelu a simulačních experimentů jsou hledány optimální hranice pro nově příchozí a stávající klienty a to pro specifický počet zaměstnanců ordinace. 
\newline
Smyslem projektu je demonstrovat zvládnutelné kapacity pro jednotlivá složení ordinace tak, aby nalezené výsledky umožňovaly řízení náboru nových zaměstnanců při potřebě navýšit počet klientů buď při řešení krizové situace z důvodu nedostatku stomatologických pracovišť v okolí nebo plánovaného rozšiřování popřípadě vzniku nového stomatologického pracoviště a reagování na očekávanou poptávku. 
\newline
Práce vznikla jako projekt do předmětu Modelování a simulace. 

\subsection{Řešitelé}

Na projektu se podíleli Lukáš Drahník a Ladislav Bezděčík. Při tvorbě byly čerpány informace z výukových materiálů k předmětu IMS, z ročenek z webových stránek České stomatologické komory \cite{ceskastomatologickakomorarocenky} a informací z pracoviště Ortodoncie a Stomatologie s.r.o. Jindřichův Hradec \cite{soldanova} a MDDr. Lukáš Novák, s.r.o. Jihlava \cite{lukasnovak}.

\subsection{Ověřování validity modelu}
Z množství získaných informací byly ponechány shodné a vytvořen obecný abstraktní model (viz. \cite{ims}, slajd 42-45) nepopisující žádné konkrétní pracoviště v celé jeho šíři. Každé pracoviště je unikát a procesy na pracovišti přizpůsobuje svému oboru působnosti, možnostem a také spektru svých klientů.
\newline
Validita (viz. \cite{ims}, slajd 37) navrhovaného modelu byla průběžně experimentálně ověřována. Toto ověřování bylo prováděno srovnávaním výstupů simulace s informacemi poskytnutými Českou stomatologickou komorou \cite{ceskastomatologickakomorarocenky} a konkrétními pracovišti viz. \cite{lukasnovak} a \cite{soldanova} a v neposlední řadě ze zkušeností v pozici klienta nejenom na již zmíněných pracovištích. Z důvodu unikátnosti každého pracoviště se předpokládá, že vypracování projektu na základě zakázky by validita modelu byla ověřována pro danou jednu konkrétní ordinaci dokud by experimenty plně neodpovídaly poskytnutým datům z jejího předchozího fungování.

\section{Rozbor tématu a použitých metod/technologií}

Při modelování (viz. \cite{ims}, slajd 8) stomatologické ordinace jsme postupovali z následujících předpokladů. Zaměstnanci jsou na pracovišti rozděleni na sestry a zubaře. Vyhnuli jsme se použití sekretářky z důvodu splnění vybraného zadání, kterým je stomatologická ordinace nikoliv stomatologické centrum. Dále jsme do systému uvedli počet křesel a rentgen, který se používá při vstupní prohlídce.
\newline
Sestřičky mají za úkol vyřizovat hovory s novými pacienty, se stávajícími pacienty, obsluhovat při vstupní prohlídce rentgen, pacienta připravit na volné křeslo pro zubaře, asistovat zubaři při pravidelné kontrole a lehkém zákroku při počtu 1 sester a těžkém zákroku při počtu 2 sester, kdy druhá se neúčastní nachystání pacienta a posléze placení, ale asistuje pouze na zákrok.
\newline
Zubaři se věnují pacientovi pouze na křesle a to při pravidelné kontrole, lehkému zákroku a těžkému zákroku. Po každém pacientovi mají čas na uklidnění pro sebe a to nejméně na dobu trvající připravení nového pacienta sestřičkou. Je nutno podotknout, že doba trvání jednotlivých zákroků záleží na zkušenostech daného zubaře a asistující sestřičky.
\newline
Každá ordinace má hranicí pacientů, které může reálně zvládnout. V systému jako kapacita registrací. Také je potřeba znát kapacitu plánovaných návštěv pro dlouhodobě udržitelný chod ordinace při udržení kvality poskytující péče.
\newline
Pacienti se stejně jako se stávají z nových pacientů stávajícími tak se ze stávajících bývalými. Je větší pravděpodobnost, že se tak stane při krátké spolupráci, případně vůbec před začátkem spolupráce než po několikaleté. 
\newline
Největší riziko ztráty nového pacienta je při objednání se na vstupní prohlídku na kterou část také vůbec nepřijde. 
\newline
U stávajícího pacienta je větší šance na ztrátu klienta při nenabídnutí termínu objednání z důvodu nedostatečné kapacity než bez ničího přičinění, např. stěhování, smrt atp. Také je zde riziko, že klient nepřijde na domluvený termín, přičemž většina se přeobjedná, část se již nikdy neozve.

\subsection{Použité postupy pro vytváření modelu}

Byl použit jazyk C++ společně s knihovnou SIMLIB \cite{simlib}, která splňovala veškerá naše očekávaní kladená zadáním a která se potvrdila použitím knihovny při vypracování. Při vypracování bylo čerpáno z prvního demonstračního cvičení \cite{dem01}, druhého demonstračního cvičení\cite{dem02} a semináři o projektech \cite{seminar} k předmětu IMS.


\section{Koncepce}

Cílem projektu je simulovat stomatologickou ordinace s konkrétním cílem zkoumat udržitelný počet klintů. Byla zanedbána poruchovost rentgenu z důvodu vysoké spolehlivosti a poruchovost křesla z důvodu neovlivňování chodu ordinace. V případě času stráveném prováděním zákroků byla hodnota odhadnuta pro průměrného zubaře, protože dle rozložení pravděpodobnosti v získaných datem z ročenek \cite{ceskastomatologickakomorarocenky} je velký počet starších zubařů mezi 60-69, stejně jako velký počet mladých 20-34, v případě, že nerozlišujeme pohlaví.

\subsection{Formy konceptuálního modelu}

Abstraktní model (viz. \cite{ims}, slajd 42 - 44) ordinace byl popsán pomocí Petriho sítě (viz. \cite{ims}, slajd 123 - 135) na základě relevantních údajů zmíněných výše. K tomuto účelu byl zvolen vhodný program podporující Petriho sítě - Pipe2 \cite{pipe2}.

\subsection{Architektura simulačního modelu}

Na začátku je generátor typu Event (viz. \cite{ims}, slajd 162), který vytváří nové pacienty. První nový pacient je vygenerovaný při spuštění simulací a poté vždy s nastaveným časového rozložení s takovým středem, který je ovlivnitelný u každého experimentu pomocí argumentů.
\newline
Všichni zaměstnanci, tzn. sestřičky, zubaři, křesla, kapacita registrací a kapacita návštěv jsou modelovány jako Store (viz. \cite{ims}, slajd 184), jejichž hodnoty lze ovlivnit u každého experimentu pomocí argumentů.
\newline
Rentgen je modelován jako Facility (viz. \cite{ims}, slajd 146), předpokládá se, že z finančních důvodů bude rentgen jenom 1.
\newline
K simulaci času vycházející již z návrhu popsaného Petriho síťí je použita funkce Exponential (viz. \cite{ims}, slajd 106).
\newline
Na přechody s procentuálním rozdělením je použita metoda Random (viz. \cite{ims}, slajd 167).
\newline
Histogram (viz. \cite{ims}, slajd 181) je použit na čas strávený v čekárně čekající na sestřičku a rentgen, čas strávený v čekárně čekající na zubaře a na sledování sestřiček.
\newline
Hlavní část programu používá Proces (viz. \cite{ims}, slajd 172), tato část modeluje chování od objednání nového pacienta, kdy volá sestřiče, po příchodu do čekárny na rentgen, který provádí sestřička, vyplnění registrace již bez sestřičky, následuje jeho první prohlídka, případné zaplacení a zařazení do role stávajících pacientů, kteří se objednávají po určité době sami pokud se tedy neobjedná na konci návštěvy.
\newline
Jako Process je také naimplementována pauza zubaře, který je po zákroku.
\newline
Do souboru společně s daty z každého Store se vypisují důležité informace pro experimenty jako například odmítnutí pacienti.

\section{Podstata simulačních experimentů a jejich průběh}

Experimenty bylo ověřováno, kolik pacientů zvládne daný počet zubařů, sestřiček s daným počtem křesel. Experimenty začínáme od nově založené ordinace, která disponuje 2 sestřičkami, 1 zubařem, 1 křeslem a inkrementujeme přičemž vypouštíme nesmyslné kombinace jako 2 zubaři a 1 křeslo atp. Jako výstup experimentů se očekává maximum možných registrací a kapacita plánovaných návštěv. Jako kritérium kdy je experiment úspěšný a my jsme nalezli přibližnou mezní hodnotu jsou sledovány obě čekárny, počet odmítnutých pacientů a počet stávajících pacientů.

\subsection{Postup experimentování}

V každém z experimentů se budeme snažit nalézt hodnoty kapacit plánovaných návštěv a maximum možných registrací. Experimenty provádíme při běhu simulace po dobu 5-ti let.\newline

 1. nastavení doby na 5 let a postupně použijeme kombinace 1 nový pacient / den, 2, 4, 8

 2. nastavení počtu sestřiček, zubařů, křesel dle varianty experimentu

 3. nastavení defaultní hodnoty pro kapacitu plánovaných návštěv a maximum možných registrací

 4. nastavení výstupního souboru rozdílného pro každý experiment

 5. spuštění simulace

 6. zkontrolování hodnot 

 7. opakování od kroku 3. upravením hodnoty kapacitu plánovaných návštěv pokud je pravda něco z následujícího:

	7.1 je příliš malá/velká vytíženost sestřiček

	7.2 je příliš malá/velká vytíženost zubařů

	7.3 dost pacientů není objednáno z důvodu malé kapacity / kapacita je využívána minimálně

	7.4 čekací doba v čekárnách je příliš velká

	7.5 stávající počet klientů je mnohem nižší než počet registrovaných

 8. opakování od kroku 3. upravením hodnoty maxima možných registrací pokud je pravda něco z následujícího:

	8.1 dost nových pacientů není objednáno z důvodu malé kapacity

	8.2 stávající počet klientů je mnohem nižší než počet registrovaných

 9. při nalezení vhodných hodnot maxim začínáme od kroku 1. s upravenou hodnotou generování nových pacientů

\subsection{Jednotlivé experimenty}
\subsubsection{Experiment 1}

Experiment probíhá při nastavení hodnot na: 1 zubař, 2 sestřičky, 1 křeslo.

\begin{table}[h!]
\centering
\begin{tabular}{|l|l|l|}
\hline
Počet nových pacientů / 24 hodin & Maximum možných registrací & Kapacita plánovaných návštěv \\ \hline
1        & 0                	& 0           		\\ \hline
2       & 0          		& 0           		\\ \hline
4       & 0   			& 0        		\\ \hline
8       & 0   			& 0           		\\ \hline
\end{tabular}
\caption {Výsledné hodnoty pro experiment 1} \label{tab:title} 
\end{table}

\subsubsection{Experiment 2}

Experiment probíhá při nastavení hodnot na: 2 zubaři, 2 sestřičky, 2 křesla.

\begin{table}[h!]
\centering
\begin{tabular}{|l|l|l|}
\hline
Počet nových pacientů / 24 hodin & Maximum možných registrací & Kapacita plánovaných návštěv \\ \hline
1        & 0                	& 0           		\\ \hline
2       & 0          		& 0           		\\ \hline
4       & 0   			& 0        		\\ \hline
8       & 0   			& 0           		\\ \hline
\end{tabular}
\caption {Výsledné hodnoty pro experiment 2} \label{tab:title} 
\end{table}

\subsubsection{Experiment 3}

Experiment probíhá při nastavení hodnot na: 2 zubaři, 3 sestřičky, 2 křesla.

\begin{table}[h!]
\centering
\begin{tabular}{|l|l|l|}
\hline
Počet nových pacientů / 24 hodin & Maximum možných registrací & Kapacita plánovaných návštěv \\ \hline
1        & 0                	& 0           		\\ \hline
2       & 0          		& 0           		\\ \hline
4       & 0   			& 0        		\\ \hline
8       & 0   			& 0           		\\ \hline
\end{tabular}
\caption {Výsledné hodnoty pro experiment 3} \label{tab:title} 
\end{table}

\subsubsection{Experiment 4}

Experiment probíhá při nastavení hodnot na: 3 zubaři, 3 sestřičky, 3 křesla.

\begin{table}[h!]
\centering
\begin{tabular}{|l|l|l|}
\hline
Počet nových pacientů / 24 hodin & Maximum možných registrací & Kapacita plánovaných návštěv \\ \hline
1        & 0                	& 0           		\\ \hline
2       & 0          		& 0           		\\ \hline
4       & 0   			& 0        		\\ \hline
8       & 0   			& 0           		\\ \hline
\end{tabular}
\caption {Výsledné hodnoty pro experiment 4} \label{tab:title} 
\end{table}

\subsubsection{Experiment 5}

Experiment probíhá při nastavení hodnot na: 3 zubaři, 5 sestřičky, 3 křesla.

\begin{table}[h!]
\centering
\begin{tabular}{|l|l|l|}
\hline
Počet nových pacientů / 24 hodin & Maximum možných registrací & Kapacita plánovaných návštěv \\ \hline
1        & 0                	& 0           		\\ \hline
2       & 0          		& 0           		\\ \hline
4       & 0   			& 0        		\\ \hline
8       & 0   			& 0           		\\ \hline
\end{tabular}
\caption {Výsledné hodnoty pro experiment 5} \label{tab:title} 
\end{table}

\subsection{Závěry experimentů}

Celkem bylo provedeno 20 experimentů. Všechny experimenty byly zaměřeny na zjištění optimálního maxima možných registrací a kapacitu plánovaných návštěv pro konkrétní počet zaměstnanců. Z uvedených experimentů vyplynulo, že ...
\newline
Výsledky a nalezená optimální maxima jsou dostupná v souboru Makefile pod příkazem make run a uloženy v nově vytvořených souborech experiment[1-5].out.

\section{Shrnutí simulačních experimentů a závěr}

Provedené experimenty potvrzují, že s dostatkem informací o konkrétní stomatologické ordinaci lze predikovat pomocí modelu a experimentů pomocí něho provedených její chování s dostatečnou přesností.

\newpage

\nocite{*}

%% BIBLIOGRAPHY
\bibliography{local}
\bibliographystyle{plain}

\newpage
\thispagestyle{empty}

\end{document}
%% END OF FILE
