\documentclass[a4paper,11pt,titlepage]{article}

\usepackage[utf8]{inputenc}
\usepackage[czech]{babel}
\usepackage[left=2cm,top=3cm,text={17cm,24cm}]{geometry}
\usepackage{graphicx}
\usepackage{listings}
\usepackage{url}
\usepackage{amsmath, amsthm, amssymb}

\begin{document}


\begin{titlepage}
	
	\begin{center}
		{\Huge\textsc{Vysoké učení technické v~Brně}}\\
		\medskip
		{\huge\textsc{Fakulta informačních technologií}}\\
		\vspace{\stretch{0.382}}
		{\huge Téma 1: Stomatologická ordinace}\\
		\medskip
		{\LARGE Technická zpráva k projektu do předmětu IMS}\\
		\vspace{\stretch{0.618}}
	\end{center}
	
	Ladislav Bezděčík xbezdec13@stud.fit.vutbr.cz \hfill{} 

	Drahník Lukáš xdrahn00@stud.fit.vutbr.cz \hfill{Brno, \today}
	
\end{titlepage}

\newpage

\tableofcontents

\newpage

\section{Úvod}

Práce se zabývá simulací (viz. \cite{ims}, slajd 8) modelu (viz. \cite{ims}, slajd 7) stomatologické ordinace. Na základě modelu a simulačních experimentů jsou hledány optimální hranice pro nově příchozí a stávající klienty a to pro specifický počet zaměstnanců ordinace. 
\newline
Smyslem projektu je demonstrovat zvládnutelné kapacity pro jednotlivá složení ordinace tak, aby nalezené výsledky umožňovaly řízení náboru nových zaměstnanců při potřebě navýšit počet klientů buď při řešení krizové situace z důvodu nedostatku stomatologických pracovišť v okolí nebo plánovaného rozšiřování popřípadě vzniku nového stomatologického pracoviště a reagování na očekávanou poptávku. 
\newline
Práce vznikla jako projekt do předmětu Modelování a simulace. 

\subsection{Řešitelé}

Na projektu se podíleli Lukáš Drahník a Ladislav Bezděčík. Při tvorbě byly čerpány informace z výukových materiálů k předmětu IMS, z ročenek z webových stránek České stomatologické komory \cite{ceskastomatologickakomorarocenky} a informací z pracoviště Ortodoncie a Stomatologie s.r.o. Jindřichův Hradec \cite{soldanova} a MDDr. Lukáš Novák, s.r.o. Jihlava \cite{lukasnovak}.

\subsection{Ověřování validity modelu}
Z množství získaných informací byly ponechány shodné a vytvořen obecný abstraktní model (viz. \cite{ims}, slajd 42-45) nepopisující žádné konkrétní pracoviště v celé jeho šíři. Každé pracoviště je unikát a procesy na pracovišti přizpůsobuje svému oboru působnosti, možnostem a také spektru svých klientů.
\newline
Validita (viz. \cite{ims}, slajd 37) navrhovaného modelu byla průběžně experimentálně ověřována. Toto ověřování bylo prováděno srovnávaním výstupů simulace s informacemi poskytnutými Českou stomatologickou komorou \cite{ceskastomatologickakomorarocenky} a konkrétními pracovišti viz. \cite{lukasnovak} a \cite{soldanova} a v neposlední řadě ze zkušeností v pozici klienta nejenom na již zmíněných pracovištích. Z důvodu unikátnosti každého pracoviště se předpokládá, že vypracování projektu na základě zakázky by validita modelu byla ověřována pro danou jednu konkrétní ordinaci dokud by experimenty plně neodpovídaly poskytnutým datům z jejího předchozího fungování.

\section{Rozbor tématu a použitých metod/technologií}

Při modelování (viz. \cite{ims}, slajd 8) stomatologické ordinace jsme postupovali z následujících předpokladů. Zaměstnanci jsou na pracovišti rozděleni na sestry a zubaře. Vyhnuli jsme se použití sekretářky z důvodu splnění vybraného zadání, kterým je stomatologická ordinace nikoliv stomatologické centrum. Dále jsme do systému uvedli počet křesel a rentgen, který se používá při vstupní prohlídce.
\newline
Sestřičky mají za úkol vyřizovat hovory s novými pacienty, se stávajícími pacienty, obsluhovat při vstupní prohlídce rentgen, pacienta připravit na volné křeslo pro zubaře, asistovat zubaři při pravidelné kontrole a lehkém zákroku při počtu 1 sester a těžkém zákroku při počtu 2 sester, kdy druhá se neúčastní nachystání pacienta a posléze placení, ale asistuje pouze na zákrok.
\newline
Zubaři se věnují pacientovi pouze na křesle a to při pravidelné kontrole, lehkému zákroku a těžkému zákroku. Po každém pacientovi mají čas na uklidnění pro sebe a to nejméně na dobu trvající připravení nového pacienta sestřičkou.
\newline
Každá ordinace má hranicí pacientů, které může reálně zvládnout. V systému jako kapacita registrací. Také je potřeba znát kapacitu plánovaných návštěv pro dlouhodobě udržitelný chod ordinace při udržení kvality poskytující péče.
\newline
Pacienti se stejně jako se stávají z nových pacientů stávajícími tak se ze stávajících bývalými. Je větší pravděpodobnost, že se tak stane při krátké spolupráci, případně vůbec před začátkem spolupráce než po několikaleté. 
\newline
Největší riziko ztráty nového pacienta je při objednání se na vstupní prohlídku na kterou část také vůbec nepřijde. 
\newline
U stávajícího pacienta je větší šance na ztrátu klienta při nenabídnutí termínu objednání z důvodu nedostatečné kapacity než bez ničího přičinění, např. stěhování, smrt atp. Také je zde riziko, že klient nepřijde na domluvený termín, přičemž většina se přeobjedná, část se již nikdy neozve.

\subsection{Použité postupy pro vytváření modelu}

Byl použit jazyk C++ společně s knihovnou SIMLIB \cite{simlib}, která splňovala veškerá naše očekávaní kladená zadáním a která se potvrdila použitím knihovny při vypracování. Při vypracování bylo čerpáno z prvního demonstračního cvičení \cite{dem01}, druhého demonstračního cvičení\cite{dem02} a semináři o projektech \cite{seminar} k předmětu IMS.


\section{Konceptuální model}



\subsection{Návrh}
\subsection{Zdroje informací}
\subsection{Validita}
\subsection{Petriho síť}
\section{Simulační model}
\subsection{Návrh}
\section{Experimenty}
\subsection{Stanovené cíle pro experimenty a způsob jejich dosažení}
\subsection{Jednotlivé experimenty}
\subsubsection{Experiment 1}
\subsubsection{Experiment 2}
\subsubsection{Experiment 3}
\subsubsection{Experiment 4}
\subsection{Závěry plynoucí z experimentů}
\section{Závěr}

\newpage

\nocite{*}

%% BIBLIOGRAPHY
\bibliography{local}
\bibliographystyle{plain}

\newpage
\thispagestyle{empty}

\end{document}
%% END OF FILE
